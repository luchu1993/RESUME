\documentclass{resume}

\usepackage{ctex}
\usepackage{float}
\usepackage{tabularx}

\author{陆\ \ 楚}
\address{上海市, 中山北路3663号, 华东师范大学}
\phone{(+86) 189-6107-3286}
\email{luchu1993@163.com}
\extras{
	Github: \href{https://github.com/luchu1993}{luchu1993}
}
\begin{document}
\maketitle

\TechnicalSkills
\begin{table}[H]
	\centering
	\begin{tabularx}{\textwidth}{X X X}
         \textbf{编程语言} & \textbf{软件}  & \textbf{机器学习}\\
		\begin{itemize}
			\item C/C++
			\item Python
			\item Qt/QML
		\end{itemize} &
		\begin{itemize}
			\item Latex
			\item Git
			\item Linux/MacOS
		\end{itemize} &
        \begin{itemize}
            \item pytorch
            \item Tensorflow/Keras
            \item sklearn
        \end{itemize}
	\end{tabularx}
\end{table}
\vspace{-4em} % Ugly hack for spacing
\Education

\School{华东师范大学\ \ \ \ 计算机科学与软件工程学院}{2019年7月}{硕士 -- 密码与网络安全}
\School{常熟理工学院\ \ \ \ 计算机科学与工程学院}{2015年7月}{本科 -- 物联网工程}


\TechnicalProjects
%----------------科研项目---------------------------
\TechnicalProject{面向用户属性的弱口令猜测技术}{2017年11月 -- 现在}
\begin{itemize}
	\item 分析了传统口令攻击算法(PCFG与Markov)的缺陷和不足
	\item 提出了一种基于深度循环神经网络(RNN)的弱口令猜测方法
    \item 结合用户群组属性,提出一种条件式的RNN的定向弱口令猜测技术
    \item 在Rockyou, CSDN, 12306等泄露口令集上进行相关实验,对比和分析实验结果
\end{itemize}

\TechnicalProject{基于生成式对抗网络的协议重构}{2017年9月 -- 2017年10月}
\begin{itemize}
	\item 分析了雾计算(Fog Computing)场景下工控协议重构的必要性
	\item 将工控协议转换成灰度图像,分析协议特性
	\item 使用深度卷积生成式对抗网络(DCGAN)实现工控协议的构建
\end{itemize}

\TechnicalProject{基于对抗样本技术的恶意软件检测攻击方法} {2017年4月 -- 2017年8月}
\begin{itemize}
	\item 分析了深度学习以及传统机器学习中面临的对抗样本的问题
	\item 提出了一种基于遗传算法(Genetic Algorithm)高效寻找对抗样本的技术
    \item 在恶意PDF文件检测系统中进行相关的实验和分析
\end{itemize}

\StudentTeams
\StudentTeam{计算机科学与软件工程学院研究生会部长}{2017年7月 -- 现在}
\begin{itemize}
	\item 组织研会活动的宣传和策划工作
	\item 组织安排学术活动
\end{itemize}

\StudentTeam{本科生程序设计课程助教}{2017年9月 -- 2018年1月}
\begin{itemize}
    \item 参与本科生计算机程序设计课程的助教工作
    \item 协助主讲老师完成学生答疑工作
\end{itemize}

\OtherWorkExperience

\Job{上海先锋商泰电子技术有限公司}{2015年1月 -- 2015年7月}{C++软件开发}
\begin{itemize}
	\item 参与车载电子系统的嵌入式软件的开发
	\item 维护相关文档,学习使用QML技术
\end{itemize}

\Job{苏州C2Fun网络科技有限公司}{2015年9月 -- 2016年3月}{游戏开发}
\begin{itemize}
	\item 参与手游《真三国无双》的开发工作
	\item 参与公司游戏引擎的优化与开发
\end{itemize}


\Intrests

\begin{itemize}
	\item 读书 -- 喜欢王小波的作品《黄金时代》,余华的《活着》,以及张悦然的《茧》
	\item 算法 -- 喜欢研究机器学习,喜欢读Bishop的经典书籍《PRML》,分享了自己的读书笔记,请参阅我的github
    \item 运动 -- 乒乓球、羽毛球
\end{itemize}

\end{document}
